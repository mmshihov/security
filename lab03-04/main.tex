\documentclass[a4paper]{article} 

\usepackage[russian]{babel}
\usepackage[utf8]{inputenc}
\usepackage{indentfirst} 
\usepackage{graphicx} 
\usepackage{amsmath} 

\title{Задания для лабораторной работы №3 (продолжительность 8 часов)}
\date{\today}
\author{Михаил Шихов \\ \texttt{\underline{m.m.shihov@gmail.com}}}

\begin{document}

\maketitle
%\tableofcontents

\begin{abstract}
    Необходимо выбрать себе задание на 3 лабораторную работу. На лекции, практике или лабораторной нужно обязательно подойти к преподавателю и \emph{сообщить номер выбранного задания}. Выбранные задания из списка доступных выбывают. В некоторых случаях, например, если используются разные языки программирования, одно и то же задание могут делать несколько групп (человек).
    
    Для сохранения истории работы над проектом \emph{необходимо} использовать систему контроля версий. При сдаче лабораторной будет проверяться история работы над проектом. Желательно использовать git (о его использовании будет рассказано на ближайших лекциях).
    
    О пометках в списке заданий. Пометка xN означает, что данный вариант может делать группа не более N человек. Пометка demo означает, что данное задание с повышенными требованиями к удобству использования (засчитывается как доклад на лекции).
    
    Выбор нужно сделать до конца марта\footnote{Этого года}. Затем задания будут назначены преподавателем. До указанного срока Вы вольны сами сформулировать интересное для Вас и преподавателя задание, обсудить и утвердить его. 
\end{abstract}

\section{Задания}

\begin{enumerate}

\item (x3) Выделить набор уникальных слов\footnote{Без лексического анализа} (словарь) из текста какого-либо художественного произведения на русском языке (на ваш выбор). Реализовать генераторы $CRC_x$ разрядности $x\in\{4,5,8,12\}$. Найти значения хешей $h=CRC_x(w)$ для каждого слова $w$ из полученного словаря. 
Для каждого варианта $CRC_x$
\begin{enumerate}
    \item Определить количество коллизий хешей для всех получившихся значений $h$. Построить гистограммы <<количество коллизий на знечение $h$>>.
    
    \item Для каждого варианта $CRC_x$ для значения хеша $h'$ с максимальным количеством коллизий вывести все слова $w$, дающие тот же хеш $h'$. 
\end{enumerate}

\item (x1) Выполнить поиск первых 10 миллионов простых чисел. 
\begin{enumerate}
    \item Найденные простые числа сохранить в текстовый файл в формате номер:число.
    \item Выделить 30 из самых длинных интервалов на множестве натуральных чисел, содержащих лишь не простые числа. Отобразить ситуацию на графике.
    \item Для каждого четного числа $x$, которое меньше последнего найденного простого найти два простых числа $p_1,p_2$, в сумме дающих $x$. Вывести в отдельный файл в формате $x=p_1+p_2$.
\end{enumerate}

\item (x2) Выполнить проектирование архитектуры защищенной социальной сети. Безопасность данных пользователя в такой сети должна зависеть только от пользователя.

\item (x2,demo) Реализовать максимально наглядный эмулятор работы RAID-контроллеров уровней 0,1,2,3,4,5. Визуализировать поэтапное прохождение обращений чтения-записи.

\item (x1) Выполнить частотный анилиз байт исходного файла, содержащего текст до и после шифрования алгоритмом:
\begin{enumerate}
    \item AES;
    \item DES.
\end{enumerate}
Можно (нужно!) взять готовую реализацию алгоритма шифрования.

\item (x2) Создать административный скрипт WMI, остлеживающий запуск определенных программ в определенное время (список программ и интервалы времени задаются в конфигурационном файле). При запуске программы скрипт делает снимок с веб-камеры и отправляет его по почте с соответствующими комментариями (кто запустил, когда, какую программу, на какой машине). 

\item (x3) Реализовать клиент-серверный сетевой модуль, остлеживающий создание, удаление, модификацию файлов в определенной папке (список папок задается в текстовом конфигурационном файле). Созданный или модифицированный файл распределяется по нескольким машинам в сети, на которых уставновлена серверная часть. Оригинал файла хранится на аутентифицируемой машине в открытом виде, дубли на других машинах зашифрованы. Допустим единый центр управления. При удалении файла его копии сохраняются некоторое время (настраиваемый параметр). При потере данных, на любой машине, где установлен модуль, можно восстановить данные, пройдя соответствующую аутентификацию.

\item (x3, demo)Демо-пример для демонстрации организации поточных шифров, основанных на LFSR (сдвиговый регистр с линейной обратной связью). Необходимо эргономичное решение для возможности создания и настройки отдельных LFSR, а также создание блоков передаточных функций (выражение функции вводится в инфиксной форме). Режимы работы: ввод ключа ползователем, шифрование файла, дешифрование файла, генерация гаммы в файл, оценки периода гаммы, выдача периода гаммы (если возможно - макс длина - параметр). Реализовать выбор подходящих вариантов обратных связей для регистров заданного периода.

\item (x2) Организовать передачу файла по зашифрованному каналу связи (по сети). Использовать стандартные алгоритмы. Комбинированная схема. Организовать генерацию пары для асссиметричной схемы. Сгенерировать ключ для симметричной схемы. Зашифровать данные симметричным ключом, затем этот ключ зашифровать на открытом ключе пары. Выполнить передачу зашифрованных данных и симметричного ключа одним сообщением по сети. 
\begin{enumerate}
    \item RSA/AES;
    \item RSA/DES3;
    \item ECRSA/AES;
    \item ECRSA/DES3;
\end{enumerate}
Можно (нужно!) взять готовую реализацию алгоритмов шифрования.

\item (x2, demo) Демо-пример для пороговой схемы разделения секрета в классической постановке. Имеется M ученых, которые совместно изобрели некоторое устройство, к которому мир еще не готов. Ученые решают запереть изобретение до момента X в надежном сейфе. Также они понимают, что в момент X они возможно не смогут собраться в полном составе. Они решают, что если они соберутся в составе не менее N человек (порог), то они смогут открыть сейф, группа же из меньшего количества ученых сейф открыть не сможет. Сколько замков нужно врезать в сейф? Сколько ключей и от каких замков нужно выдать каждому ученому? Кроме корректной работы важна качественная визуализация, возможность обойтись только мышью в процессе показа (выбор ученых, выбор из них пороговой группы), наглядный процесс генерации сейфа и раздачи ключей. Наглядная демонстрация попыток открыть сейф группой меньше пороговой (демонстрация нехватки ключей) и попыток открыть сейф группой с числом ученых не менее N.


\item (x2) Реализовать вирус исходного кода (интерпретируемые языки). Вирус внедряется в программу, выводит сообщение, находит потенциальный файл, проверяет заражен ли он, и заражает, если не заражен. Начать работу должен троянский конь (уже зараженный скрипт). Распространение только \emph{вглубь} каталога, в котором находится троянский конь. Также написать антивирус. Язык:
\begin{enumerate}
    \item PERL;
    \item PHP;
    \item PYTHON;
    \item Java Script (windows shell);
    \item Visual Basic (windows shell).
\end{enumerate}


\item (x2) Реализовать вирус исходного кода (компилируемые языки). Вирус внедряется в программу, выводит сообщение, находит потенциальный файл, проверяет заражен ли он, и заражает, если не заражен. Начать работу должен скомпилированный троянский конь (уже зараженный на уровне исходного текста). Распространение только \emph{вглубь} каталога, в котором находится троянский конь. Упрощая, предполагаем, что проект состоит из одного файла с исходным текстом. Также написать антивирус. Язык:
\begin{enumerate}
    \item C;
    \item C++;
    \item Pascal;
    \item Java.
\end{enumerate}

\item (x1) Реализовать пороговую $(n,k)$ схему разделения секрета. В случае реализации сетевого варианта: (x3)

\item (x1) Метод расслаивания изображения Наора и Шамира для разделения секрета.

\item (x1) Стеганография. Сокрытие информации в аудиофайлах.

\item (x1) Стеганография. Сокрытие информации в файлах изображений (не BMP-формат!!!).

\item (x1) Реализовать протокол аутентификации с помощью одноразовых паролей на основе хеш-функций (one-way hash chain). В качестве $H(M)$ взять $H(M)=a^M \mod p$.

\item (x3) Реализовать генераторы случайных и псевдослучайных чисел, реализовать и провести тесты на качество. Сравнить качество <<случайных>> и псевдослучайных последовательностей.

\item (x3) Реализовать систему электронного голосования.

\item (x2) Игра <<камень, ножницы, бумага>>. Сетевой вариант. Исключить возможность обмана с помощью залоговой схемы

\item (x3, demo) Игра <<телепат>>: игрокам раздается $N$ карточек с изображениями. Передатчик выбирает наугад карточку и <<задумывает>> её. Приемник <<угадывает>> задуманную картинку. Если карточка угадана, то приемнику начисляется число очков, равное числу карт, из которого делался выбор. Использованная карта удаляется из игры, раунды повторяются до тех пор, пока не останется одна карта. Сетевой вариант. Исключить возможность обмана с помощью залоговой схемы.

\item (x2,demo) Реализовать демо-пример схемы с нулевым разглашением на основе изоморфизма графов.

\item Реализовать протокол распределения сеансовых ключей (сетевой вариант с доверенным лицом):
\begin{enumerate}
    \item (x2) Широкоротой Лягушки. Алгоритм шифрования --- AES;
    \item (x2) Нидхейма-Шредера.  Алгоритм шифрования --- DES3;
    \item (x2) Отвэй-Риса.  Алгоритм шифрования --- AES;
\end{enumerate}
Можно (нужно!) взять готовую реализацию алгоритма шифрования.

\item Реализовать протокол аутентификации на основе ассиметричной схемы (сетевой вариант):
\begin{enumerate}
    \item (x2) Алгоритм цифровой подписи --- DSA;
    \item (x2) Алгоритм цифровой подписи --- ECDSA;
\end{enumerate}
Можно (нужно!) взять готовую реализацию алгоритма цифровой подписи.

\item (x1,x2) Выполнить рефакторинг кода, переписав код с языка C на C++. Задания подбираются индивидуально.

\item (x1) Написать программу, определяющую, подписан ли PE32+ модуль.

\end{enumerate}

\end{document}
