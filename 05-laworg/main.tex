\usepackage{etex} %эта магическая херь избавляет от переполнения регистров TeX а!!!

\mode<article>{\usepackage{fullpage}}
\mode<presentation>{
    \usetheme{Madrid}
    \useoutertheme{shadow}
} 

\usepackage[utf8]{inputenc}
\usepackage[russian]{babel}
\usepackage{indentfirst}
\usepackage{graphicx}

\usepackage{amsmath}
\usepackage{amsfonts}
\usepackage{amsthm}
%\usepackage{algorithm}
%\usepackage{algorithmic}

%\usepackage[all]{xy}

\date{Лекция по дисциплине <<методы и средства защиты компьютерной информации>> (\today)}
\author[М.~М.~Шихов]{Михаил Шихов \\ \texttt{\underline{m.m.shihov@gmail.com}}}

%%для рисования графов пакетом xy-pic
%\entrymodifiers={++[o][F-]}

%%для псевдокода алгоритмов (algorithm,algorithmic)
%\renewcommand{\algorithmicrequire}{\textbf{Вход:}}
%\renewcommand{\algorithmicensure}{\textbf{Выход:}}
%\renewcommand{\algorithmiccomment}[1]{// #1}
%\floatname{algorithm}{Псевдокод}

%\setbeamercolor{alerted text}{fg=-green} %gyan, blue, green, -green



\title[Методы защиты информации]{Методы защиты информации}


\begin{document}


%титул и содержание статьи
\mode<article>{\maketitle\tableofcontents}

%титул и содержание презентации
\frame<presentation>{\titlepage}
\begin{frame}<presentation>[allowframebreaks]
\frametitle{Содержание}
\tableofcontents
\end{frame}


\section{Методы защиты информации}

\begin{frame}
\frametitle{Методы защиты информации}
\begin{itemize}
    \item Правовые;
    \item Организационные;
    \item Технические.
\end{itemize}
\end{frame}

%TODO
%правовые, организационные, технические
%структура АИС


\subsection{Правовые методы}


\begin{frame}
\frametitle{Правовые методы}
\framesubtitle{Гарантии доступности}
Конституция РФ:

\alert{Ст.29,п.4:}Каждый имеет право свободно искать, получать, передавать, производить и распространять информацию любым законным способом. Перечень сведений, составляющих государственную тайну, определяется федеральным законом.

\alert{Ст.42,п.4:}Каждый имеет право на благоприятную окружающую среду, достоверную информацию о ее состоянии и на возмещение ущерба, причиненного его здоровью или имуществу экологическим правонарушением
\end{frame}


\begin{frame}
\frametitle{Правовые методы}
\framesubtitle{Гарантии целостности}
\begin{itemize}
    \item конституция;
    \item кодексы: гражданский, трудовой, об административных правонарушениях, трудовой;
    \item федеральные законы: 1-ФЗ от 2002 г. <<Об электронной цифровой подписи>>.
\end{itemize}
\end{frame}


\begin{frame}
\frametitle{Правовые методы}
\framesubtitle{Гарантии конфиденциальности. Категории ограниченного доступа}
К информации ограниченного доступа относят категории:
\begin{itemize}
    \item государственная тайна;
    \item персональные данные;
    \item коммерческая тайна;
    \item секрет производства (ноу-хау, know-how), в том числе результат интеллектуальной деятельности;
    \item банковская тайна;
    \item информация, составляющая кредитную историю.
\end{itemize}
В конституции: \alert{Ст.24:}Сбор, хранение, использование и распространение информации о частной жизни лица без его согласия не допускаются.
\end{frame}


\begin{frame}
\frametitle{Правовые методы}
\framesubtitle{Гарантии конфиденциальности. Правовое поле}

\begin{itemize}
    \item Конституция РФ.
    \item Кодексы.
    \begin{itemize}
        \item Уголовный кодекс РФ.
        \item Кодекс об административных нарушениях.
        \item Гражданский кодекс (часть 2, часть 4).
        \item Трудовой кодекс.
    \end{itemize}
    \item Законы.
    \begin{itemize}
        \item №149-ФЗ: <<Об информации, информационных технологиях и о защите информации>>.
        \item №98-ФЗ: <<О коммерческой тайне>>.
        \item №395-ФЗ: <<О банках и банковской деятельности>>.
        \item №218-ФЗ: <<О кредитных историях>>.
        \item №152-ФЗ: <<О персональных данных>>.
    \end{itemize}
\end{itemize}
\end{frame}


%TODO подробнее об 149-м законе

Закон №149-ФЗ: <<Об информации, информационных технологиях и о защите информации>> устанавливает восемь принципов регулирования отношений, возникающих в сфере информации, информационных технологий и защиты информации:
\begin{enumerate}
    \item свобода поиска, получения, передачи, производства и распространения информации любым законным способом;
    \item установление ограничений доступа к информации только федеральными законами;
    \item открытость информации о деятельности государственных органов и органов местного самоуправления и свободный доступ к такой информации, кроме случаев, установленных федеральными законами;
    \item равноправие языков народов Российской Федерации при создании информационных систем и их эксплуатации;
    \item обеспечение безопасности Российской Федерации при создании информационных систем, их эксплуатации и защите содержащейся в них информации;
    \item достоверность информации и своевременность ее предоставления;
    \item неприкосновенность частной жизни, недопустимость сбора, хранения, использования и распространения информации о частной жизни лица без его согласия;
    \item недопустимость установления нормативными правовыми актами каких-либо преимуществ применения одних информационных технологий перед другими, если только обязательность применения определенных информационных технологий для создания и эксплуатации государственных информационных систем не установлена федеральными законами.
\end{enumerate}



\subsection{Организационные методы}


\begin{frame}
\frametitle{Организационные методы}
\begin{definition}%theorem, lemma, proof, corollary, example
\alert{Организационная защита информации} --- составная часть системы защиты информации, определяющая и вырабатвывающая порядок и правила функционирования объектов защиты и деятельности должностных лиц в целях обеспечения защиты информации.
\end{definition}
Выделяют следующие принципы организационной защиты:
\begin{itemize}
    \item принцип комплексного подхода --- эффективное использование сил, средств, способов и методов защиты информации в зависимости от складывающейся ситуации и наличия угроз;
    \item принцип оперативности принятия управленцеских решений;
    \item принцип персональной ответственности.
\end{itemize}
\end{frame}


\begin{frame}
\frametitle{Направления организационной защиты}
\begin{definition}%theorem, lemma, proof, corollary, example
    \alert{Сведения} --- запечатленные в организме результаты отражения движения объектов материального мира
\end{definition}
\begin{itemize}
    \item Организация внутреннего и пропускного режимов и охраны.
    \item Организация аналитической работы.
    \item Организация работы с персоналом.
    \item Комплексное планирование мероприятий по защите информации.
    \item Организация работы с носителями сведений.
\end{itemize}
\end{frame}


\begin{frame}
\frametitle{Задачи, решаемые организационными методами}
\begin{itemize}
    \item Реализация на предприятии механизма управления, обеспецивающего защиту конфиденциальной информации.
    \item Осуществление принципа персональной ответственности за защиту КИ.
    \item Определение перечня сведений, относимых к различным категориям конфиденциальной информации.
    \item Ограничение круга лиц, имеющих доступ к информации различных грифов секретности.
    \item Подбор и обучение персонала.
    \item Организация и ведение конфиденциального делопроизводства.
    \item Систематический контроль за соблюдением установленных требований по защите информации.
\end{itemize}
\end{frame}


Ведущую роль в организации защиты информации играют на предприятии его руководитель и заместитель руководителя. Должность: специалист по защите информации, администратор.

\begin{frame}
\frametitle{Основные виды структурных подразделений}
\begin{itemize}
    \item Режимно-секретное (служба безопасности, служба защиты информации на предприятиях, где отсутствует режим государственной тайны).
    \item Подразделение по технической защите информации и противодействию иностранным техническим разведкам.
    \item Подразделение криптографической защиты.
    \item Подразделение охраны и пропускного режима.
    \item Мобилизационное подразделение (в условиях военного времени).
\end{itemize}
\end{frame}


\subsection{Технические методы}


\begin{frame}
\frametitle{Инженерно-техническая защита}
\begin{definition}%theorem, lemma, proof, corollary, example
    \alert{Инженерно-техническая защита} --- совокупность специальных мер, персонала и технических средств, обеспечивающая информационную безопасность.
\end{definition}
Методы технической защиты могут быть разделены:
\begin{itemize}
    \item по объектам защиты; %TODO: узлы, каналы связи, 
    \item по характеру мероприятий; %TODO экранирование, подавление шумов, преграды
    \item по масштабу охвата; %TODO уровня предприятия, работника, инф системы
    \item по классам технических средств защиты; %TODO см. далее...
    \item по классам технических средств нападения.
\end{itemize}
\end{frame}


\begin{frame}
\frametitle{Инженерно-техническая защита}
\frametitle{Технические средства защиты}
\begin{itemize}
    \item Физические.
    \begin{itemize}
        \item Системы ограждения и физической изоляции.
        \item Системы контроля физического доступа.
        \item Запирающие устройства и хранилища.
    \end{itemize}
    \mode<article>{
        Устройства, сооружения и меры, затрудняющие проникновение злоумышленника.
    }
    
    \item Аппаратные.
    \mode<article>{
        Устройства (механические, электрические, электронные и др.) предназначенные для защиты информации.
    }
    
    \item Программные.
    \mode<article>{
        Программные средства решают задачи аутентификации пользователей, систем и аппаратуры, проверка легитимности доступа к данным, аудит, сигнализация о НСД, уничтожение побочной информации после работы программ (очистка памяти). 
    }
    
    \item Криптографические.
    \mode<article>{
        Аппаратные и программные средства для защиты на уровне \emph{представления} информации.
    }
    
    \item Комбинированные.
    
\end{itemize}
\end{frame}


\appendix %приложения


\section{УК РФ}


\subsection{Статья 146}

\begin{frame}[allowframebreaks]
    \frametitle{Статья 146}
    \framesubtitle{Нарушение авторских и смежных прав}
    \begin{enumerate}
        \item Присвоение авторства (плагиат), если это деяние причинило крупный ущерб автору или иному правообладателю, наказывается штрафом в размере до двухсот тысяч рублей или в размере заработной платы или иного дохода осужденного за период до восемнадцати месяцев, либо обязательными работами на срок до четырехсот восьмидесяти часов, либо исправительными работами на срок до одного года, либо арестом на срок до шести месяцев.
        
        \item Незаконное использование объектов авторского права или смежных прав, а равно приобретение, хранение, перевозка контрафактных экземпляров произведений или фонограмм в целях сбыта, совершенные в крупном размере, наказываются штрафом в размере до двухсот тысяч рублей или в размере заработной платы или иного дохода осужденного за период до восемнадцати месяцев, либо обязательными работами на срок до четырехсот восьмидесяти часов, либо исправительными работами на срок до двух лет, либо принудительными работами на срок до двух лет, либо лишением свободы на тот же срок.
        
        \item Деяния, предусмотренные частью второй настоящей статьи, если они совершены:
        \begin{enumerate}
            \item группой лиц по предварительному сговору или организованной группой;
            \item в особо крупном размере;
            \item лицом с использованием своего служебного положения,
        \end{enumerate}
        наказываются принудительными работами на срок до пяти лет либо лишением свободы на срок до шести лет со штрафом в размере до пятисот тысяч рублей или в размере заработной платы или иного дохода осужденного за период до трех лет или без такового.
    \end{enumerate}
    Деяния, предусмотренные настоящей статьей, признаются совершенными в крупном размере, если стоимость экземпляров произведений или фонограмм либо стоимость прав на использование объектов авторского права и смежных прав превышают сто тысяч рублей, а в особо крупном размере - один миллион рублей.
\end{frame}

 
\subsection{Статья 272}

\begin{frame}[allowframebreaks]
    \frametitle{Статья 272}
    \framesubtitle{Неправомерный доступ к компьютерной информации}
    \begin{enumerate}
        \item Неправомерный доступ к охраняемой законом компьютерной информации, если это деяние повлекло уничтожение, блокирование, модификацию либо копирование компьютерной информации, наказывается штрафом в размере до двухсот тысяч рублей или в размере заработной платы или иного дохода осужденного за период до восемнадцати месяцев, либо исправительными работами на срок до одного года, либо ограничением свободы на срок до двух лет, либо принудительными работами на срок до двух лет, либо лишением свободы на тот же срок.
        
        \item То же деяние, причинившее крупный ущерб или совершенное из корыстной заинтересованности, наказывается штрафом в размере от ста тысяч до трехсот тысяч рублей или в размере заработной платы или иного дохода осужденного за период от одного года до двух лет, либо исправительными работами на срок от одного года до двух лет, либо ограничением свободы на срок до четырех лет, либо принудительными работами на срок до четырех лет, либо арестом на срок до шести месяцев, либо лишением свободы на тот же срок.
        
        \item Деяния, предусмотренные частями первой или второй настоящей статьи, совершенные группой лиц по предварительному сговору или организованной группой либо лицом с использованием своего служебного положения, наказываются штрафом в размере до пятисот тысяч рублей или в размере заработной платы или иного дохода осужденного за период до трех лет с лишением права занимать определенные должности или заниматься определенной деятельностью на срок до трех лет, либо ограничением свободы на срок до четырех лет, либо принудительными работами на срок до пяти лет, либо лишением свободы на тот же срок.
        
        \item Деяния, предусмотренные частями первой, второй или третьей настоящей статьи, если они повлекли тяжкие последствия или создали угрозу их наступления, наказываются лишением свободы на срок до семи лет.
    \end{enumerate}
    Крупным ущербом в статьях настоящей главы признается ущерб, сумма которого превышает один миллион рублей.
\end{frame}
 

\subsection{Статья 273}

\begin{frame}[allowframebreaks]
    \frametitle{Статья 273}
    \framesubtitle{Создание, использование и распространение вредоносных компьютерных программ}
    \begin{enumerate}
        \item Создание, распространение или использование компьютерных программ либо иной компьютерной информации, заведомо предназначенных для несанкционированного уничтожения, блокирования, модификации, копирования компьютерной информации или нейтрализации средств защиты компьютерной информации, наказываются ограничением свободы на срок до четырех лет, либо принудительными работами на срок до четырех лет, либо лишением свободы на тот же срок со штрафом в размере до двухсот тысяч рублей или в размере заработной платы или иного дохода осужденного за период до восемнадцати месяцев.
        
        \item Деяния, предусмотренные частью первой настоящей статьи, совершенные группой лиц по предварительному сговору или организованной группой либо лицом с использованием своего служебного положения, а равно причинившие крупный ущерб или совершенные из корыстной заинтересованности, наказываются ограничением свободы на срок до четырех лет, либо принудительными работами на срок до пяти лет с лишением права занимать определенные должности или заниматься определенной деятельностью на срок до трех лет или без такового, либо лишением свободы на срок до пяти лет со штрафом в размере от ста тысяч до двухсот тысяч рублей или в размере заработной платы или иного дохода осужденного за период от двух до трех лет или без такового и с лишением права занимать определенные должности или заниматься определенной деятельностью на срок до трех лет или без такового.
        
        \item Деяния, предусмотренные частями первой или второй настоящей статьи, если они повлекли тяжкие последствия или создали угрозу их наступления, наказываются лишением свободы на срок до семи лет.
    \end{enumerate}
\end{frame}
 
 
\subsection{Статья 274}

\begin{frame}[allowframebreaks]
    \frametitle{Статья 274}
    \framesubtitle{Нарушение правил эксплуатации средств хранения, обработки или передачи компьютерной информации и информационно-телекоммуникационных сетей}
    \begin{enumerate}
        \item Нарушение правил эксплуатации средств хранения, обработки или передачи охраняемой компьютерной информации либо информационно-телекоммуникационных сетей и оконечного оборудования, а также правил доступа к информационно-телекоммуникационным сетям, повлекшее уничтожение, блокирование, модификацию либо копирование компьютерной информации, причинившее крупный ущерб, наказывается штрафом в размере до пятисот тысяч рублей или в размере заработной платы или иного дохода осужденного за период до восемнадцати месяцев, либо исправительными работами на срок от шести месяцев до одного года, либо ограничением свободы на срок до двух лет, либо принудительными работами на срок до двух лет, либо лишением свободы на тот же срок.
        
        \item Деяние, предусмотренное частью первой настоящей статьи, если оно повлекло тяжкие последствия или создало угрозу их наступления, наказывается принудительными работами на срок до пяти лет либо лишением свободы на тот же срок.
    \end{enumerate}
\end{frame}


\section{Источники}
%сводка по ссылкам
\begin{frame}
    \frametitle{Источники}
    Организационные методы подробно обсуждаются в \cite{bib:romanov:org}. Методы защиты в совокупности представлены в \cite{bib:yaroch:infsec}. 
    
    Настоятельно рекомендуется изучить приведенные федеральные законы в их актуальной редакции.
\end{frame}


\begin{frame}[allowframebreaks]{Библиография}
    \bibliographystyle{gost780u}
    \bibliography{./../bibliobase}
\end{frame}

\end{document}